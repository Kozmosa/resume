% !TEX program = xelatex
% This file is generated, don't manually edit!
\documentclass{resume}
\usepackage{lastpage}
\usepackage{fancyhdr}
\usepackage{linespacing_fix}
\usepackage[fallback]{xeCJK}
\renewcommand\headrulewidth{0pt}

\begin{document}

\name{roife}

\basicInfo{
  % \phone{(+86)} \textperiodcentered\
  \email{roifewu at gmail dot com} \textperiodcentered\
  \github[roife]{https://github.com/roife} \textperiodcentered\
  \homepage[roife.github.io]{https://roife.github.io}
}

\section{教育背景}
\educationentry{北京航空航天大学}{本科}{2019.09 -- 2023.06}{
  计算机科学与技术专业 | GPA: 3.84/4.00
}

\section{科研经历}
\researchEntry{面向深度学习神经网络算子的轻量级端侧编译器}{北京航空航天大学}{2022.06 -- 2023.06}{
  \begin{itemize}
    \item 本科毕业设计项目,包括独立实现的轻量级端侧编译器和对LLVM代码生成模块的裁剪工作
    \item 利用端侧的 shape 信息对离线编译的LLVM IR格式神经网络算子进行二次优化,以降低算子运行时的时空开销
  \end{itemize}
}

\section{项目开发}
\projectEntry{Ayame}{Java, ARMv7 | 合作,主要负责寄存器分配与 codegen 相关优化}{https://github.com/No-SF-Work/ayame}{
  \begin{itemize}
    \item C 子集编译器,使用 SSA IR,可导出 LLVM-IR/ARMv7,实现了 GVN、图着色寄存器分配等优化;
    \item 系统能力大赛编译系统设计赛第二名,在比赛近 1/3 的测试点上性能超过 \texttt{clang -O3}。
  \end{itemize}
}

\projectEntry{Racoon}{Rust, LLVM IR}{https://github.com/roife/racoon}{
  Rust 实现的 SysY - LLVM IR 编译器;北航软件学院 19 级编译课设的 Rust 版参考实现。
}

\projectEntry{航概 Hanggai}{SwiftUI, Vue, Rails | 合作,主要负责 iOS 以及 Web 后端开发}{https://github.com/Caniformia}{
  北航《航空航天概论》课程刷题应用,支持 Web 端与 iOS 端;「航概」已上架 AppStore。
}

\section{奖项荣誉}
\scholarshipEntry{国家奖学金}{}{2022.09}
\scholarshipEntry{全国大学生计算机系统能力大赛·编译系统设计赛全国总决赛}{一等奖,第二名}{2021.08}

\section{专业技能}
\begin{itemize}
  \entry{编程语言}{
    C, C++, Java, Rust, Swift, Python, JavaScript, Ruby, Verilog(SV), Haskell
  }
  \entry{技术栈}{
    Vue, Rails, Django, SwiftUI | Docker, PostgreSQL, Redis, CI, Git
  }
  \entry{编译器}{
    熟悉 LLVM IR,了解基于 SSA 的优化算法,对 LLVM 的架构有一定了解
  }
  \entry{程序语言理论}{
    熟悉函数式编程范式,了解类型系统相关知识,学习过定理证明器 Coq 和 Agda
  }
  \entry{开发环境}{
    macOS, Linux | Emacs, Xcode, JetBrains IDEs
  }
\end{itemize}

\section{其他}
\begin{itemize}
  \entry{助教工作} {
    \begin{itemize}
    \datedEntry{程序设计基础训练}{北航信息类助教}{2020.09 - 2021.02}
      {出题/答疑/监考等}
    \datedEntry{面向对象设计与构造}{北航计算机学院 S.T.A.R 团队}{2021.09 - 2022.06}
      {系统开发/系统运维/出题/答疑/监考等}
    \end{itemize}
  }
  \entry{博客}{
    \url{https://roife.github.io}
    }
  \entry{外语}{
    汉语(母语),英语(CET-6)
    }
\end{itemize}

\end{document}
